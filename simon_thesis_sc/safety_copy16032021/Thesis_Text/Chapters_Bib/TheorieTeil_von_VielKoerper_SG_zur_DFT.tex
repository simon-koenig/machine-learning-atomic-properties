\chapter{TheorieTeil von VielK"orper SG zur DFT}

\section{Vielk"orperproblematik der analytischen QM, Grundlegend f"ur Ein-K"orper Problem}
Die L"osung der Schr"odingergleichung stellt eines der fundamentalsten Probleme der Nicht-Klassischen Physik dar. In seiner einfachsten zeitunabh"angigen Form wird sie angeschrieben als 
$$TODO: write easy SG e.q$$
mit: 
$$TODO: Define parameter of simple SG eq.$$
Die Schr"odingergleichung ist als Eigenwertproblem definiert, somit ist ein Set an Basiswellenfunktionen zu finden welche f"ur einen entsprechenden Hamilton-Operator eine Energie liefern die das System beschreibt. Die Wahl des Hamiltonoperators ist systemspezifisch hat aber im allgemeinen immer die Form: 
$$TODO: H=T+V$$
Durch den Hamiltonian wird also eine Energie beschrieben, die immer aus der Kinetischen Energie TODO:make italicT des Systems und aus der Potenziellen Energie TODO:make italic V des Systems besteht.


Betrachtet man beispielsweise den Fall des einfachsten Atoms, des TODO: Wassserstoff-Atoms,  l"asst sich die Gleichung exakt l"osen. Symmetrie"uberlegungen f"uhren zu einer Betrachtung der "Bewegung zweier Teilchen  im kugelsymmetrischen Feld. Wie in der klassischen Mechanik l"asst sich das 2-K"orperproblem miteinander wechselwirkender Teilchen in der Quantenmechanik auf ein 1-K"orperproblem reduzieren." \footcite[6]{akt1}. Da das Proton im Kern eine viel gr"oßere Masse hat als das Elektron TODO: eminus, wird die Bewegung des Protons vernachl"assigt. Dies ist die Kernaussage der \textit{Born-Oppenheimer-N"aherung}, welche auch bei Vielteilchenproblemen vorteilhaft ist und auf die im Weiteren noch genauer eingegangen wird. Somit zerf"allt die Wellenfunktion, in die freie Bewegung des Massenmittelpunktes von Proton und Elektron und in die Relativbewegung des Elektrons selbst. Die Bewegung des Massenmittelpunktes im freien Raum wird vernachl"assigt und f"ur das Elektron l"asst sich der Hamiltonian in der Form

$$TODO:Hamiltonian Wasserstoff f"ur eminus$$

anschreiben. 

Und die gesamte Schr"odingergleichung in kartesischen Koordinaten

$$TODO: SG f"ur Wasserstoff$$

mit den Termen: 

$$TODO: TERME erkl"aren von SG Wasserstoff kartesisch$$

Nach einer Transformation auf Kugelkoordinaten nutzt man die Kugelsymmetrie des Feldes aus. Die Erhaltung des Drehimpulses im kugelsymmetrischen Feld erlaubt  eine Separation der Eigenfunktion in einen winkelabh"angigen Teil und eine Radialkomponente. \footcite[vgl.][218]{physik4}

$$TODO: Separierte Wellenfunktion in radial und Winkelteil$$

Die L"osungen der Winkelkomponente des Wellenfunktion sind die Kugelfunktionen \footcite[vgl.][222]{physik4}

$$TODO: Kugelfunktionen anschreiben$$

und die L"osung des Radialteils ergeben sich zu: \footcite[vgl.][230]{physik4}

$$TODO: Radialfuntkion anschreiben$$

F"ugt man den Produktansatz wieder zusammen und l"ost mit dem Hamiltonian des Wasserstoffs Atom das Eigenwertproblem  so erh"alt man f"ur jedes Energieniveau $n^2$-Eigenfunktionen. Jedes Energieniveau ist also $n^2$-fach entartet. Konkret ergeben sich die diskreten Eigenzust"ande, abh"angig von der Hauptquantenzahl n, zu: \footcite[vgl.][234]{physik4}


$$TODO: Diskrete Eigenzust"ande$$

Die Eigenfunktionen werden durch die Hauptquantenzahl $n$, die Nebenquantenzahl $l$, und die magnetische Quantenzahl $m_l$ charakterisiert und bilden eine Orthogonalsystem. 

$$TODO: zeige Eigenfunktinen sind Orthogonalysystem$$

Die realste und wichtigste Eigenschaft der Wellenfunktionen wird erst bei der Messung eines quantenmechanischen Zustandes messbar. Die Aufenthaltswahrscheinlichkeit das elektron im Volumenelement $dV$ anzutreffen ist gegeben durch: \footcite[vgl.][234]{physik4}

$$TODO: Formel f"ur aufenthalswahrscheinlichkeit wie in physik 4$$





%%%%%%%%%%%%%%%%%%%%%%.   KapitelEnde.   %%%%%%%%%%%%%%%%%%%%%%%%%%%%%

\section{Vielk""orperProblematik, Diskussion der Hartree-Fock Methode als "ahnlicher, aber doch verschiedener Ansatz + Born Oppenheimer N"aherung als wichtige Beitr"age zur DFT}

\subsection{Vereinfachungen hervorstreichen}

Im Vorangegangenen Abschnitt wurde die Schr"odingergleichung erfolgreich f"ur ein sehr vereinfachtes System analytisch exakt gel"ost und Energieeigenwerte $E_n$gefunden. Zur Simplizit"at des Systems kommen noch das Nicht-Ber"ucksichtigen von experimentalisch festgestellten Ph"anomenen wie dem Spin oder Vakuumfluktuationen. Der Spin f"uhrt außerdem direkt ein magnetisches Moment sich, welches demzufolge auch außen vorgelassen wird. Weitere vereinfachende Annahmen sind: 
\begin{itemize}
\item Elektron wird als nicht relativistisches Teilchen betrachtet
\item Wechselwirkung zwischen Elektron und Proton wird als Coulomb-Wechselwirkung angenommen
\item Proton befindet sich in Ruhe 
\end{itemize}


\subsection{Vielk"orperproblem der QM}
Der bisher verwendete Hamiltonian und die zugeh"origen Eigenfunktionen beschreiben ein sehr simples System. Reale Problemstellungen beziehen sich auf gr"oßere Systeme, sowohl von Elektronen als auch Kernen. Schon einfache Molek"ule besitzen eine Vielzahl an Teilchen und sind mit analytischer Mathematik nicht mehr l"osbar. Als Vorstufen zur Dichtefunktionaltheorie gilt es noch weitere Approximation zu ber"ucksichtigen. Die im folgenden verwendeten Methoden finden sich in den "uberlegungen zu DFT wieder. 

Zwei notwendige Approximationen  beziehungsweise Ans"atze stellen die 

\begin{itemize}
\item Born-Oppenheimer N"aherung, und die 
\item Hartree- bzw. Hartree-Fock Methode
\end{itemize}
dar. 

In atomare Einheiten kann der, ein System aus $N_n$  Kernen und $N_e$  Elektronen beschreibendem, Vielk"orper Hamiltonian geschrieben werden als: \footcite[Lecture 4][2]{hande}


$$TODO: schreibe vielkorper hamiltonian an, sowie wie hande ihn anschreibt$$

Mit der Einheitentransformation auf atomare Einheiten: 

$$TODO: Einheitentransformation auf atomare Einheiten$$

\subsubsection{Born-Oppenheimer N"aherung}
Betrachtet man den Operator der kinetischen Energie $TODO: Operator T$ in einem Massenschwerpunktsystem mit einem Nukleus und N Elektronen, "ahnlich dem des Wasserstoffatoms. Im Allgemeinen Fall gibt es nicht ein $e^-$ sondern N Elektronen. 

$$TODO: Operator der Kinetischen Energie ohne Born Oppenheimer N"aherung$$

Born und Oppenheimer schlagen vor, aufgrund der großen Massendifferenz von Elektron und Proton, die Bewegung des Protons gegen"uber der des Kerns zu vernachl"assigen. Dem zugrunde liegt die Idee, dass sich Elektronen aufgrund ihrer um den Faktor $10^3$ kleineren Masse viel schneller bewegen als die schweren Kerne. Veranschaulicht bedeuetet das, die Elektronen sind bei jeder Bewegung der Kerne schon wieder in ihren Grundzustand relaxiert. Aufgrund dessen kann die Bewegung des Kerns vernachl"assigt werden und gehen nich mehr in den Hamiltonian mit ein. Somit vereinfacht sich der Operator der kinetischen Energie im Limes $TODO lim M gegen unendlich$. 
\footcite[272]{hjorth-jensen}

$$TODO: Operator der Kinetischen Energie MIT Born Oppenheimer N"aherung$$


Die Born-Oppenheimer N"aherung wird eingef"uhrt f"ur ein System aus einem Nukleus und N Elektronen. Ihr Konzept ist umlegbar auf einen allgemeinen Hamiltonian mit  $N_n$  Kernen und $N_e$  Elektronen gem"aß TODO:gleichungsreferenz. Die Born-Oppenheimer N"aherung geht in die Dichtefunktionaltheorie instrumental ein. 


\subsubsection{Hartree-bzw. Hartree Fock Methode}
Die Hartree Methode, und im weiterf"uhrenden die Hartree Fock Methode macht einen Ansatz f"ur die Vielelektron Wellenfunktionen. Die Terme des allgemeinen Vielteilchenhamiltonians aus (TODO: Gleichung f"ur  VielTeil.Hamiltonian) welche sich auf Kern-Kern Wechselwirkung beziehen werden als konstant angenommen. Diese Annahme ist durch die Born-Oppenheimer N""aherung begr"undet.


Der Elektronen-Hamilton Operator l"asst sich schreiben als: 

$$TODO: schreibe elektronen hamiltonian an, sowie wie hande ihn anschreibt$$

Die ersten beiden Terme sind Summen "uber Ein-Teilchen Operatoren und beziehen sich nur auf eine Elektron-Koordinate. Diese Beitr"age zur Energie sind trivial, da sie  "aquivalent sind zu L"osungen der  Einteilchen-Schr"odingergleichungen f"ur die jeweiligen Elektronen, anschließend wird "uber diese einzelnen Energeibeitr"age summiert.  Die Schr"odingergleichung separiert hier, weil jeder Ein-Teilchen Operator nur eine Wellenfunktion "\textit{sieht}" , welche nur abh"angig von der Koordinate des jeweiligen Teilchens ist.\footcite[Lecture 4][1]{hande}

Der interessante Term ist demfolgend:

$$TODO: schreibe elektronen hamiltonian an, sowie wie hande ihn anschreibt aber nur den letzten Term "uber die zwei Summen $$

Aus dieser Gleichung liest sich das Potential f"ur das \textit{i-te} Elektron, hier wieder mit dem Elektron-Kern Potential, wie folgt:

$$Potential f"ur i-tes Atom wie auf Seite 77 in akt1 skript$$

Hier wird ein Problem sichtbar, welches ebenso in der Dichtefunktionaltheorie auftaucht. Das Potential im Hamiltonian, dass ja zur Bestimmung der Eigenfunktionen notwendig ist, h"angt von ebenen jenen Eigenfunktionen selbst ab. Die N"aherung von Hartree kann als "Zentralfeldn"aherung" bezeichnet werden, da sie jeweils das  Potentials des  \textit{i-te} Elelektrons als Summe "uber die Wahrscheinlichkeitsdichten aller anderen Elektronen modelliert. Eine "ahnliche "uberlegung findet sich in der DFT wieder. \footcite[78]{akt1}



Hartree verwendet als L"osungsansatz das direkte Produkt der Elektronenorbitale. Dieser Ansatz ber"ucksichtigt das Pauli Prinzip nicht.

$$ Hartree Ansatz f"ur Einteilchen-Produkt Wellenfunktion$$

Da nun Elektronen Fermionen sind und deshalb bei einer Symmetrievertauschung, formal der Anwendung des Parit"atsoperators, einer Vorzeichen"anderung unterliegen, gen"ugt dieser einfache Produktansatz nicht. Eine Verbesserung legen \textit{Hartreee \& Fock} dar, und zwar die Ersetzung der Orbitalprodukte durch Slaterdeterminanten. Die Darstellung der Wellenfunktion durch Slaterdeterminanten welche der Antisymmetrie-Bedingung  gen"ugt erf"ullt das Ein-Teilchen Bild.  \footcite[Lecture 4][2]{hande}

$$Write Slater-Determinant like hande$$

Die L"osung ist aufgrund der direkten Abh"angigkeit des Hamiltonians von den durch ihn zu bestimmenden Eigenfunktionen auch f"ur eine g"unstige Wahl der Eigenfunktionen nicht eindeutig l"usbar. Eine h"aufig verwendete L"osungsmethode ist die des \textit{Selbst-Konsistenten Feldes}. Die nummerische L"osung setzt einen gut gew"ahlten ersten Ansatz der Wellenfunktionen voraus und optimiert diese dann gem"aß einer iterativen Methode: 


\begin{enumerate}
	\item Ansatz f"ur Eigenfunktionen, verschiedene Methoden m"oglich (z.B.: Thomas-Fermi-Gleichung)
	\item Potentiale werden bestimmt mittels Aufenthaltswahrscheinlichkeit: $TODO Aufenthalswahrscheinlichkeit$
	\item Neue Eigenfunktionen werden mit dem gegebenen Potential bestimmt 
\end{enumerate}

Der Vorgang wird iterativ numerisch gel"ost bis sich eine \textit{selbstkonsistente} L"osung einstellt. 


%%%%%%%%%%%%%%%%%%%%%%.   KapitelEnde.   %%%%%%%%%%%%%%%%%%%%%%%%%%%%%



\section{Ansatz der DFT mit allen Termen}
\subsection{Warum DFT?}
Die L"osung der Schr"odingergleichung ist durch die Approximationen von Born-Oppenheimer beziehungsweise Hartree-Fock einfacher geworden.
Hartree-Fock separieren die Eigenfunktionen eines N-Teilchen Systems in ein Produkt aus N-Orbitalen f"ur jeweils eine Raumkoordinate. Dieser Ansatz braucht f"ur ein Elektron nun 3 Raumdimensionen, f"ur zwei Elektronen 6 Raumkoordinaten und f"ur N Elektronen $3*N$Raumkoordinaten. Dieser Ansatz bringt bereits eine Reduktion des Rechenaufwandes im Gegensatz zu den $3^N$ Raumdimensionen, die f"ur eine exakte L"osung der N-dimensionalen Schr"odingergleichung notwendig sind. F"ur reale Vielk"orperprobleme sind immer noch Kalkulationen von extremem Aufwand notwendig. Ein simples $CO_2$ Atom besitzt 22 Elektronen (TODO: Grafik einf"ugen), die Schr"odingergleichung wird ein 66-dimensionales Problem. 

$$TODO: CO2 Atom anschreiben = 22elektronen anschreiben$$

Die Konfiguration eines Blei Einheitszelle, mit etwa 100 Atomen, ergibt bei 14 Elektronen pro Pb-Atom, ein 42 000-dimensionales Problem. 
$$TODO: Silizium Einheitszelle Anschreiben = 42 000 elektronen$$

Mit Quanten-Chemie Methoden wie zum Beispiel der \textit{Møller–Plesset perturbation theory} betr"agt die Rechendauer um 
ein derartiges Problem, wie das der Blei Einheitszelle, zu l"osen, zirka ein Jahr. Mithilfe der \textit{Dichtefunktionaltheorie} ist 
eine L"osung in etwa 5 Stunden realistisch. (TODO: zitiere https://ocw.mit.edu/courses/materials-science-and-engineering/
3-021j-introduction-to-modeling-and-simulation-spring-2012/part-ii-lectures-videos-and-notes/lecture-3/)


\subsection{Annahmen der DFT, Energie als Funktional der Dichte}

Walter Kohn schreibt in seinem Artikel, zu Ehren seiner Nobelpreisverleihung, die der Dichtfunktionaltheorie zugrundeliegende Aussage: \footcite[7]{nobel-kohn}

"The basic lemma of HK. The ground-state density n(r) of a bound system of interacting electrons in some external potential v ( r ) determines this potential uniquely (Hohenberg and Kohn, 1964)"

Ist die Energie eines Systems nur von der Elektronendichte abh"angig, bedeutet das, f"ur eine beliebig große Elektronenkonfiguration, eine Reduktion auf 3 Raumkoordinaten. 
Der Hamilton-Operator eines Systems, in der Born-Oppenheimer-N"aherung, l"asst sich schreiben als: 

$$TODO: Schreibe H=T+Vext+V mit Dichte als n wie in FK2$$
$$TODO: beschreibe diese Terme wie in Hutter$$
Die Theorie von Kohn, Hohenberg und Sham geht auf das Jahr 1964 zur"uck und fußt auf zwei fundamentalen Theoremen: 


\begin{enumerate}
\item $V_{ext} $ ist bis auf eine Konstante eindeutig als Funktional der Elektronendichte $TODO:n$ bestimmt.
\item Das Variationsprinzip belegt dieses Theorem eindeutig. (TODO: maybe Beweis skizzieren)

\item  Das Funktional $V_{ext}$ hat sein Minimum gegen"uber einer Variation $TODO:variation-teilchendichte$ der Teilchendichte f"ur eine Gleichgewichtsdichte $n_0$ f"ur gegebenes $V_{ext} $. 
\item Das zweite Theorem ist ebenfalls eine Folge des allgemeinen Variationsprinzips der Quantenmechanik (TODO: maybe Beweis skizzieren)

\end{enumerate}

Allgemein kann die Energie auf Basis der Hohenberg-Kohn-Shams-Theoreme (HKS) f"ur ein System aus N wechselwirkenden Teilchen im externen Potential $V_{ext} $ geschrieben werden als\footcite[6]{dft-hutter} 

$$TODO:schreibe E von n = 4 Terme wie bei dft-hutter seite 6 ganz unten$$

$$TODO:Erkl"are Terme$$


Zun"achst betrachtet man den einfacheren Fall von N nicht-wechselwirkende Teilchen. Der Hamiltonian ist gegeben als:

$$TODO: Hamiltonian ohne V_ee Wechselwirkung$$

Die Eigenfunktionen der Schr"odinger-Gleichung

$$TODO: SG f"ur nicht-wechselwirkendes-System$$

in einem Nicht-Welchselwirkenden-System k"onnen mittels Produktansatz gel"ost werden und sind, aufgrund der Natur der Elektronen als Fermionen, durch die Slater-Determinanten der einzelnen Orbitale gegeben. Die Elektronendichte ist dann:

$$TODO: elektronendichte anschreiben als summe "uber eigenfunktionen$$

Die Kinetische Energie $T_s$  als Funktional der Dichte $TODO elektronendichte n$ ist dann: 

$$TODO: Funktional von T wie dft-hutter seite 6$$

Im allgemeinen Fall, wie in der Gleichung (TODO: reference zu E von n = 4) beschrieben, wechselwirken die Elektronen untereinander. Addiert und subtrahiert man das bereits bestimmte $T_s$ von der Gleichung  (TODO: reference zu E von n = 4) ergibt sich \footcite[7]{dft-hutter}

$$TODO: Gleichung f"ur E_0 aber mit E_{XC} wie dft-hutter seite 7 oben$$

mit dem Austausch-Korrelationspotential: 

$$TODO: Gleichung f"ur E_{XC} wie dft-hutter seite 7 fast oben$$

Die Grundzustandsenergie als Funktional der Elektronendichte besteht nun aus l"osbaren Termen und dem Austausch-Korrelationspotential $E_{XC}$. Um die Dichte zu kalkulieren werden die Kohn-Sham-Funktionen ben"otigt,  diese sind Eigenfunktionen der Kohn-Sham-Gleichungen \footcite[161]{fk2}

$$TODO: Kohn-Sham-Gleichung wie in FK2 Seite 161 oben, mit V = V_{ext}$$

 
mit

$$TODO: m"u = Variaton von E_XC, FK2 Seite 160$$


Die Kohn-Sham-Gleichungen sind verallgemeinerte Ein-Teilchen-Schr"odingergleichung in denen ein effektives Potential $V_{eff}$ anstatt eines Vielteilchenpotentials angeschrieben wird. Sie ber"ucksichtigen das Austausch-Korrelationspotential $E_{XC}$. 

Die Lagrange-Multiplikatoren $\epsilon_i$ sind orthogonalit"atserhaltend f"ur die Kohn-Sham-Orbitale. Bei der Kalkultation der Eigenfunktionen der Kohn-Sham Gleichung tritt ein "ahnliches Problem zu dem der Hartree-Fock Methode auf: Der Kohn-Sham-Hamiltonian ist direkt von der Elektronendichte abh"angig. Zur numerischen L"osung der Kohn-Sham-Gleichungen und somit zum finden der Elektronendichte kann die Methode des Selbst-Konsistenten-Feldes, wie bei Hartree-Fock, verwendet werden. 

Die Kohn-Sham-Gleichungen mit ihren iterativen L"osungsmethoden stellen die Grundlage von \textit{ab-initio} Methoden dar, zu denen auch die in dieser Arbeit verwendete Methode VASP(Vienna Ab Initio Package) geh"ort. \footcite[161]{fk2}





\section{L"osung der DFT}


