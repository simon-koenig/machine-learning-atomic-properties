\chapter{Theory DFT}
\label{chapter:2}
%%%%%%%%%%%%%%%%%%%%%% %%%%%%%%%%%%%%%%%%%%%%%%%%%%%%%%%%%%%%%%%%%%%%%%%%% %%%%%%%%%%%%%%%%%%%%%%%%%%%%%


\section{Many Body Problem of Quantum Theory}
\label{section:2.1}

Solving the \textit{Schrodinger Equation} is one of the fundamental problems of modern physics. The time independent equation is given in \ref{eq:SE_simple}. 
The \textit{Schrodinger Equation} is defined as Eigenvalue problem, demanding a set of \textit{wave-functions} as solution. Applying the Hamiltonian on a set of \textit{eigen-wave-functions} yields an associated energy. 

\begin{equation}
	 \hat{H}\ket{\Psi} = E \ket{\Psi} 
	\label{eq:SE_simple}
\end{equation}

\begin{itemize}
	\item $ \hat{H }$ ... Hamiltionian
	\item $ \ket{\Psi} $ ... Basis-Wave-Function in Dirac Notation, eigenfunction of the \textit{Schrodinger Equation}
	\item E ... Energy of the system, eigenvalue of the \textit{Schrodinger Equation}
\end{itemize}



The \textit{Hamiltonian} is the quantum-mechanical analogy to the classical concept of mechanical energy. Equation \ref{eq:HTV} states that the \textit{Hamiltonian} is the sum of the \textit{Kinetic Energy T} and the \textit{Potential Energy V}, here denoted as Operators. 


\begin{equation}
	 \hat{H} = \hat{T}+\hat{V}
	\label{eq:HTV}
\end{equation}


Solving the \textit{Eigenvalue Problem} analytically is possible for a small number of objects, for example one electron or an \textit{Helium Atom}.
Real problems relate to large systems with a large amount of objects. The solution of the \textit{Schrodinger Equation} becomes unsolvable analytically and numeric methods come into place. The numerical method to solve the \textit{Many Body Schrodinger Equation} in \ref{label:H_manybody} used in this work is the \textit{Density Functional Theory (DFT)}. 

In atomic units, a system of $N_n$ nuclei and $N_e$ electron is described by the  \textit{Many Body Schrodinger Hamiltonian} in \footcite[Lecture 4][2]{hande}: 

\begin{equation}
	\hat{H} = -\frac{1}{2} \sum_{i}^{N_e}\nabla_i^2
- 
 \sum_{i}^{N_e} 
\sum_{I}^{N_n}
\frac{Z_I}{\lvert \vec{r_i}-\vec{R_I} \rvert}
 + 
 \frac{1}{2} 
 \sum_{i}^{N_e} 
 \sum_{i\neq j }^{N_e}
 \frac{Z_I}{\lvert \vec{r_i}-\vec{r_j} \rvert}
+ 
 \frac{1}{2} 
 \sum_{I}^{N_n} 
 \sum_{I\neq J}^{N_n}
 \frac{Z_I Z_J}{\lvert \vec{R_i}-\vec{R_j} \rvert}
 	
	\caption{Hamiltonian of Many Body Schrodinger Equation for $N_n$ nuclei and $N_e$ electrons }
	\label{eq:H_manybody}
\end{equation}


\begin{itemize}
	\item $ \hat{H }$ ... Hamiltonian
	\item $r_i$, $r_j$  ... position of $i_{th}$, $j_{th}$ electron
	\item $R_i$, $R_j$  ... position of $i_{th}$, $j_{th}$ nucleus
	\item $Z_i$, $Z_j$  ... charge of $i_{th}$, $j_{th}$ nucleus
\end{itemize}



As precursors to the \textit{DFT} two approximations are discussed int the following. The ideas of the presented methods offer an approach to finding a numerical solution to the \textit{Many Body Schrodinger Equation}  in \ref{label:H_manybody}  and lay the foundation for \textit{DFT}. 

The necessary approximations are: 

\begin{itemize}
	\item Born-Oppenheimer Approximation 
	\item Hartree and Hartree-Fock Approximations 
\end{itemize}


\subsection{Born Oppenheimer Approximation}
\label{subsection:2.1.1}


%%%%%%%%%%%%%%%%%%%%%% %%%%%%%%%%%%%%%%%%%%%%%%%%%%%%%%%%%%%%%%%%%%%%%%%%% %%%%%%%%%%%%%%%%%%%%%%%%%%%%%
\subsection{Born Oppenheimer Approximation}
\label{subsection:2.1.1}

Born and Oppenheimer suggest, because of the significant mass difference of electron and nucleus to neglect the movement of the nuclues in respect to the electron.s The underlying idea is that the electrons, because of their mass being lower than that of the nucleus by a magnitude of $10^3$, relax to an equilibrium state rapidly for any movement of the nucleus. Any movement of the nucleus does not affect the energy of the electrons as it is assumed the $e^-$ move to equilibrium state instantly. Details to the idea of Born and Oppenheimer are to be found in \footcite[272]{hjorth-jensen}.

The Kinetic Energy Operator $\hat{T}$ of a \textit{center-of-momentum-frame} with one nucleus and N electrons in equation \ref{eq:T_withoutBorn} before the Born-Oppenheimer Approximation is reduced to \ref{eq:T_withBorn}  after 
the idea of Born and Oppenheimer is applied. Mathematically the idea can be expressed by performing the limit M $ \to\infty$.

\begin{equation}
	  \hat{T} = -\frac{\hbar^2}{2(M+Nm)} \nabla_{CM}^2
- 
\frac{\hbar}{2\mu}
\sum_{i}^{N} 
\nabla_{i}^2
-
\frac{\hbar^2}{M}
 \sum_{i > j}^{N} 
  \nabla_i  \nabla_j

	\label{eq:T_withoutBorn}
\end{equation}


\begin{itemize}
	\item $ \hat{H }$ ... Kinetic Energy
	\item M ... mass of nucleus
	\item m ... mass of electrons
	\item N ... Number of electrons
	\item \nabla_{CM} ... nabla of the center of mass 
	\item \nabla_i, \nabla_j nabla, of $i_{th}$, $j_{th}$ electron
	\item \mu = $\frac{Mm}{M+m}$ ... reduced mass 
\end{itemize}

\begin{equation}
	  \hat{T} =
- 
\frac{\hbar}{2m}
\sum_{i}^{N} 
\nabla_{i}^2

	\label{eq:T_withBorn}
\end{equation}

The concept is introduced for a system of one nucleus and N electrons and it then expanded to any arbitrary Hamiltonian describing  $N_n$ nuclei and $N_e$ electrons in \ref{eq:H_manybody}. The Born-Oppenheimer approximation is one of the main ideas building and instrumental to \textit{Density Functional Theory}.


\subsection{Hartree and Hartree-Fock methods}
\label{subsection:2.1.2}

Hartree and in the following Hartee and Fock make an ansatz for the \textit{Many Electron Wave Function}. The last term, describing the nuclei-nuclei interaction, of the \textit{Many Body Hamiltonian} in \ref{eq:H_manybody} is assumed to be constant with the Born Oppenheimer Approximation. The \textit{Many Body Hamiltonian} is reduced to \ref{eq:H_withoutNN}. 



\begin{equation}
	\hat{H} = 
-
\frac{1}{2} \sum_{i}^{N_e}\nabla_i^2
- 
 \sum_{i}^{N_e} 
\sum_{I}^{N_n}
\frac{Z_I}{\lvert \vec{r_i}-\vec{R_I} \rvert}
 + 
 \frac{1}{2} 
 \sum_{i}^{N_e} 
 \sum_{i\neq j }^{N_e}
 \frac{Z_I}{\lvert \vec{r_i}-\vec{r_j} \rvert}

 	
	\caption{Hamiltonian of Many Body Schrodinger Equation for $N_n$ nuclei and $N_e$ electrons without nuclei-nuclei interaction}
	\label{eq:H_withoutNN}
\end{equation}


The first two terms of \ref{eq:H_manybody} are trivial since they are equivalent to the solution of the \textit{One Body Schrodinger Equation}. Solving the eigenvalue problem for the first and the second term leads to a separable differential equation. The first term in  \ref{eq:H_manybody} is the Kinetic Energy $\hat{T}_e$ for the $i_{th}$ electron and the second the term is the Potential Energy $\hat{V}_en$ for the $i_{th}$ electron in respect to the $I_{th}$ nucleus. The solution for the eigenfunctions here is a product of the wave functions, solving the \textit{One Body Schrodinger Equation}. \footcite[Lecture 4][1]{hande}


The third term represents the Potential Energy $\hat{V}_ee$ and the denominator term entangles the coordinates of the $i_{th}$ and the $j_{th}$ electron. 



\begin{equation}
	\hat{V}_{ee} = 
	 \frac{1}{2} 
 \sum_{i}^{N_e} 
 \sum_{i\neq j}^{N_e}
 \frac{1}{\lvert \vec{r_i}-\vec{r_j} \rvert}
 	
	\caption{Potential Energy for electron-electron interaction }
	\label{eq:H_Vee}
\end{equation}

The differential equation is not separable for \ref{eq:H_Vee}, this is where \textit{Hartrees} approximation applies. The main idea is that the potential applying to any elctron is identical for every electron of the system and is caused by all the other electrons and nuclei. An ansatz for $V_{ee}$ is made in \ref{V_Hartree} with the potential being proportional to the proximity density of the wave functions $\psi_i$. This is called the \textit{Mean-Field-Approximation}.

\begin{equation}
  \hat{V}_{ee}(\vec{r}_i) = 
 \sum_{i\neq j}^{N_e}
 \frac{\lvert  \psi_{ij} \rvert^2}{\lvert \vec{r_i}-\vec{r_j} \rvert} 	
	\caption{ Mean-Field-Approximation}
	\label{eq:V_Hartree}
\end{equation}

This approximation leads to a problem in computing the potential, thus also the Hamiltonian of the \textit{One Body Schrodinger Equation}. The solution of the partial eigenvalue function in \ref{eq_SE_Vee} is self dependent. The Potential $V_{ee} on the left side of the equation depends on the solution of the equation - the wave functions $ \psi_i $.  \footcite[77]{akt1}


\begin{equation}
\hat{V}_{ee}(\vec{r}_i) \psi_i 
= E \psi_i
	\caption{Partial Schrodinger Equation $V_{ee}$}
	\label{eq:SE_Vee}
\end{equation}

The same problem occurs in \textit{Density Functional Theory} and is solved by using a self-consistent method. This means looping through an algorithm by making an educated guess for the eigenfuctions then computing the potential and using the potential to compute new eigenfuctions. 

A possible initial wave function, proposed by Hartee, to solve the \textit{Partial Schrodinger Equation} in \ref{eq:SE_Vee} with a \textit{self-consistent} method is a product of all the N  \textit{Single-Electron-Wave-Functions} in \ref{eq:product_wavefunction}. \footcite[78]{akt1}


\begin{equation}
\psi_i (\vec{r}_1,...,\vec{r}_N)
 =
 \psi_{i1} (\vec{r}_1)\psi_{i2} (\vec{r}_2)...\psi_{iN} (\vec{r}_N)
	\caption{Product of Single Electron Wave Functions}
	\label{eq:product_wavefunction}
\end{equation}


The idea was later improved by Hartee and Fock, by conforming to the \textit{Pauli Principle}. Electrons are Fermions and have to apply to the \textit{Parity Operator} $\hat{P}$. The Pauli Principle demands the wave functions to be anti-symmetric. The revised solution for the wave function is a linear combination of  \textit{Slater-Determinants}, where the entries of the \textit{Slater Determinant} are product wave functions of  the \textit{Single-Electron-Wave-Functions}.   Details in \footcite[Lecture 4][2]{hande}.

Hartree and Hartree-Fock solve the \textit{Many Body Schrodinger equation}  numerically using a self-consistent method. The same concept is used in \textit{Density Functional Theory}. The \textit{self-consistent} algorithm is a iteratively repeated process. 
\begin{enumerate}
	\item Ansatz for eigen functions
	\item Potentials are determined by computing the probability density of the eigenfunctions: $ {\lvert  \psi_{ij} \rvert^2}$
	\item  New eigenfunctions are computed with the given Potential
\end{enumerate}

 The algorithm is an iterative process and is repeated until a \textit{self-consistent} solution is found. 





\section{Density Functional Theory}
\label{section:2.2}
%%%%%%%%%%%%%%%%%%%%%% %%%%%%%%%%%%%%%%%%%%%%%%%%%%%%%%%%%%%%%%%%%%%%%%%%% %%%%%%%%%%%%%%%%%%%%%%%%%%%%%
\subsection{Why DFT?}
\label{subsection:2.2.1}

The solution of the \textit{Many Body Schrodinger Equation} has become feasible after applying \textit{Hartree \&  Focks} and \textit{Born \& Oppenheimers} approximations. With Hartree the eigenfunctions  of the N-body problem separate into a product of N orbitals for every spacial coordinate. One electron equals the observation of the 3 cartesian coordinates, N electrons equal the observation of $3N$ coordinates. The \textit{Hartree & Fock} method provides a reduction in coordinates compared to the $3^N$ orbitals necessary for solving the  Schrodinger Equation exactly. For a real \textit{Many Body Problem} the computation in $3N$ coordinates is expensive. An $CO_2$ Molecule for instance, consists of 22 electrons, the related Schrodinger Equation is a 66-dimensional problem. 

TODO: maybe CO2 Molecule figure


\begin{equation}
	CO_2 = 22 electrons
	\label{eq:co2}
\end{equation}

The configuration of a lead unit cell contains about 100 Atoms. One Pb-Atom possesses 14 electrons, which counts up to 1400 electrons per unit cell. The eigenvalue problem has to be solved for 4200 dimensions. 

\begin{equation}
	Pb unic cell = 1400 electrons
	\label{eq:pb}
\end{equation}


Using \textit{Quantum Chemistry} methods, like the \textit{Møller–Plesset perturbation theory}, computation time sums up to about one year for the lead unit cell. \textit{Density Functional Theory} reduces computation time significantly and the \textit{Many Body Schrodinger Equation} for the lead unit cell can be solved in about 5 hours. Informations on this are derived from \footcite{MITlecture3}


%%%%%%%%%%%%%%%%%%%%%% %%%%%%%%%%%%%%%%%%%%%%%%%%%%%%%%%%%%%%%%%%%%%%%%%%% %%%%%%%%%%%%%%%%%%%%%%%%%%%%%
\subsection{Assumptions of DFT, Energy as Functional of Density}
\label{subsection:2.2.2}

In the article, related to the honour of his nobel price achievement, Walter Kohn writes: 

"The basic lemma of HK. The ground-state density n(r) of a bound system of interacting electrons in some external potential v ( r ) determines this potential uniquely (Hohenberg and Kohn, 1964)" \footcite[7]{nobel-kohn}

This Lemma, is the substantial concept underlying \textit{Density Functional Theory}. Kohn proposes the ground state energy of any quantum system depends solely on the electron density $n(r)$. For any electron configuration the \textit{Many Body Schrodinger Equation} is a problem in 3 coordinates. The Hamiltonian in Born-Oppenheimer Approximation is given in equation \ref{eq:Hamiltonian_DFT}. \footcite[6]{dft-hutter}

\begin{equation}
	\hat{H} = \hat{T} + \hat{V}_{ext} + \hat{V}_{ee} 
	\label{eq:Hamiltonian_DFT}
\end{equation}

\begin{itemize}
	\item \hat{T} ... Kinetic Energy of the \textit{free} electrons 
	\item  \hat{V}_{ext} ... External Potential as functional of the electron density $\vec{n}(r)$
	\item \hat{V}_{ee}  ... Potential describing the electron-electron-interaction
\end{itemize}

The ideas of \textit{DFT} proposed by \textit{Kohn, Hohenberg & Sham} in 1964 rest on two fundamental theorems: 

\begin{enumerate}
\item $V_{ext} $ ist bis auf eine Konstante eindeutig als Funktional der Elektronendichte $TODO:n$ bestimmt. \hat{V}_{ext} is, apart from a constant term, determined uniquely as a functional of the electron density of the system: $\vec{n}(r)$

\item The functional $V_{ext} $ has its Minimum in respect to a variation of the electron density  $\partial \vec{n}(r)$  for a density $ \vec{n}_0(r)$ for a given external potential  $V_{ext} $. 
\end{enumerate}



%%%%%%%%%%%%%%%%%%%%%% %%%%%%%%%%%%%%%%%%%%%%%%%%%%%%%%%%%%%%%%%%%%%%%%%%% %%%%%%%%%%%%%%%%%%%%%%%%%%%%%
\section{Solutions for DFT}
\label{subsection:2.2.3}

A general formula for the energy based on the theorems by \textit{Kohn, Hohenberg & Sham}  in equation  \ref{eq:E_HKS} as denoted by \footcite[6]{dft-hutter} 

\begin{equation}
	E_0 = E_V(n_0) = T[n_0] + \int V_{ext}(r)[n_0](r)dr + J[n_0] + E_{NC}[n_0]
	\label{eq:E_HKS}
\end{equation}


uses the terms: 

\begin{itemize}
	\item  $E_0$ ... Ground State Energy 
	\item  $n_0$ Ground State Electronic Density 
	\item  $\hat{V}_{ext}[n_0]$ ... External Potential 
	\item  $J[n_0] $ ... Classical Coulomb Energy
	\item  $E_{NC}[n_0]$ ... Non-classica electron electron Energy.
\end{itemize}

Taking a step back and having \textit{Hartree} in mind, finding the ground state energy has to be done by a self consistent method. \textit{Hartrees} idea is applied by the \textit{DFT}, the electron density is determined by the sum over the density of the probability of presence: 

$$  n(r) = \sum_{i}^{N} \lvert \psi_{i}(r) \rvert^2  $$

The \textit{DFT} introduces the concept of the Exchange Correlation Functional given by \ref{eq:Eex} with $ T_s[n] $ as the Kinetic Energy for the non-interacting electrons.

\begin{equation}
	E_{XC}[n] = T[n] - T_s[n] + E_{NC}[n]
	\caption{Exchange Correlation Energy}
	\label{eq:Eex}
\end{equation}


and the related potential by: 

\begin{equation}
	V_{XC}[n] = \frac{\delta E_{XC}[n]}{\delta n(r)}
	\caption{Exchange Correlation Potential}
	\label{eq:Vex}
\end{equation}



With the \textit{Exchange Correlation Energy} $E_{XC}[n]$, the total Energy of the system is reduced to eq \ref{eq:E_HKS_Exc} with 

\begin{equation}
	E_V(n_0) = T_s[n_0] + \int V_{ext}(r)[n_0](r)dr + J[n_0] + E_{XC}[n_0]
	\label{eq:E_HKS_Exc}
\end{equation}



The equations \ref{eq:E_HKS_Exc},  \ref{eq:Vex} and  \ref{eq:Eex} lead to the \textit{Kohn-Sham_Equations} in \ref{eq:KS} used to determine the eigenfunctions $\psi (r)$. Information is taken from \footcite[161]{fk2}.


\begin{equation}
	\left[T_s + V(r) + \int \frac{n_0(r')}{\lvert r - r' \rvert } dr' +  V_{XC}(n_0)\right] \psi_i 
= 
\epsilon_i \psi_i
	\caption{Kohn-Sham equation}
	\label{eq:KS}
\end{equation}


The solution the Kohn Sham equations are the \textit{Kohn-Sham-Orbitals} and have to be found in a \textit{self-consistent fashion}. The eigenvalue problem is a generalized \textit{One Body Schrodinger Equation}, with  \epsilon_i as langrange multiplier obtaining orthogonality for the \textit{Kohn-Sham-Orbitals}. 








%%%%%%%%%%%%%%%%%%%%%% %%%%%%%%%%%%%%%%%%%%%%%%%%%%%%%%%%%%%%%%%%%%%%%%%%% %%%%%%%%%%%%%%%%%%%%%%%%%%%%%
\subsection{Assumptions of DFT, Energy as Functional of Density}
\label{subsection:2.2.2}

Walter Kohn schreibt in seinem Artikel, zu Ehren seiner Nobelpreisverleihung, die der Dichtfunktionaltheorie zugrundeliegende Aussage: \footcite[7]{nobel-kohn}

"The basic lemma of HK. The ground-state density n(r) of a bound system of interacting electrons in some external potential v ( r ) determines this potential uniquely (Hohenberg and Kohn, 1964)"

Ist die Energie eines Systems nur von der Elektronendichte abh"angig, bedeutet das, f"ur eine beliebig große Elektronenkonfiguration, eine Reduktion auf 3 Raumkoordinaten. 
Der Hamilton-Operator eines Systems, in der Born-Oppenheimer-N"aherung, l"asst sich schreiben als: 

$$TODO: Schreibe H=T+Vext+V mit Dichte als n wie in FK2$$
$$TODO: beschreibe diese Terme wie in Hutter$$
Die Theorie von Kohn, Hohenberg und Sham geht auf das Jahr 1964 zur"uck und fußt auf zwei fundamentalen Theoremen: 


\begin{enumerate}
\item $V_{ext} $ ist bis auf eine Konstante eindeutig als Funktional der Elektronendichte $TODO:n$ bestimmt.
\item Das Variationsprinzip belegt dieses Theorem eindeutig. (TODO: maybe Beweis skizzieren)

\item Das Funktional $V_{ext}$ hat sein Minimum gegen"uber einer Variation $TODO:variation-teilchendichte$ der Teilchendichte f"ur eine Gleichgewichtsdichte $n_0$ f"ur gegebenes $V_{ext} $. 
\item Das zweite Theorem ist ebenfalls eine Folge des allgemeinen Variationsprinzips der Quantenmechanik (TODO: maybe Beweis skizzieren)

\end{enumerate}

Allgemein kann die Energie auf Basis der Hohenberg-Kohn-Shams-Theoreme (HKS) f"ur ein System aus N wechselwirkenden Teilchen im externen Potential $V_{ext} $ geschrieben werden als \footcite[6]{dft-hutter} 

$$TODO:schreibe E von n = 4 Terme wie bei dft-hutter seite 6 ganz unten$$

$$TODO:Erkl"are Terme$$


Zun"achst betrachtet man den einfacheren Fall von N nicht-wechselwirkende Teilchen. Der Hamiltonian ist gegeben als:

$$TODO: Hamiltonian ohne V_ee Wechselwirkung$$

Die Eigenfunktionen der Schr"odinger-Gleichung

$$TODO: SG f"ur nicht-wechselwirkendes-System$$

in einem Nicht-Welchselwirkenden-System k"onnen mittels Produktansatz gel"ost werden und sind, aufgrund der Natur der Elektronen als Fermionen, durch die Slater-Determinanten der einzelnen Orbitale gegeben. Die Elektronendichte ist dann:

$$TODO: elektronendichte anschreiben als summe "uber eigenfunktionen$$

Die Kinetische Energie $T_s$  als Funktional der Dichte $TODO elektronendichte n$ ist dann: 

$$TODO: Funktional von T wie dft-hutter seite 6$$

Im allgemeinen Fall, wie in der Gleichung (TODO: reference zu E von n = 4) beschrieben, wechselwirken die Elektronen untereinander. Addiert und subtrahiert man das bereits bestimmte $T_s$ von der Gleichung  (TODO: reference zu E von n = 4) ergibt sich \footcite[7]{dft-hutter}

$$TODO: Gleichung f"ur E_0 aber mit E_{XC} wie dft-hutter seite 7 oben$$

mit dem Austausch-Korrelationspotential: 

$$TODO: Gleichung f"ur E_{XC} wie dft-hutter seite 7 fast oben$$

Die Grundzustandsenergie als Funktional der Elektronendichte besteht nun aus l"osbaren Termen und dem Austausch-Korrelationspotential $E_{XC}$. Um die Dichte zu kalkulieren werden die Kohn-Sham-Funktionen ben"otigt,  diese sind Eigenfunktionen der Kohn-Sham-Gleichungen \footcite[161]{fk2}

$$TODO: Kohn-Sham-Gleichung wie in FK2 Seite 161 oben, mit V = V_{ext}$$

 
mit

$$TODO: m"u = Variaton von E_XC, FK2 Seite 160$$


Die Kohn-Sham-Gleichungen sind verallgemeinerte Ein-Teilchen-Schr"odingergleichung in denen ein effektives Potential $V_{eff}$ anstatt eines Vielteilchenpotentials angeschrieben wird. Sie ber"ucksichtigen das Austausch-Korrelationspotential $E_{XC}$. 

Die Lagrange-Multiplikatoren $\epsilon_i$ sind orthogonalit"atserhaltend f"ur die Kohn-Sham-Orbitale. Bei der Kalkultation der Eigenfunktionen der Kohn-Sham Gleichung tritt ein "ahnliches Problem zu dem der Hartree-Fock Methode auf: Der Kohn-Sham-Hamiltonian ist direkt von der Elektronendichte abh"angig. Zur numerischen L"osung der Kohn-Sham-Gleichungen und somit zum finden der Elektronendichte kann die Methode des Selbst-Konsistenten-Feldes, wie bei Hartree-Fock, verwendet werden. 

Die Kohn-Sham-Gleichungen mit ihren iterativen L"osungsmethoden stellen die Grundlage von \textit{ab-initio} Methoden dar, zu denen auch die in dieser Arbeit verwendete Methode VASP(Vienna Ab Initio Package) geh"ort. \footcite[161]{fk2}




%%%%%%%%%%%%%%%%%%%%%% %%%%%%%%%%%%%%%%%%%%%%%%%%%%%%%%%%%%%%%%%%%%%%%%%%% %%%%%%%%%%%%%%%%%%%%%%%%%%%%%
\section{L"osung der DFT}
\label{subsection:2.2.3}








% equation template

\begin{equation}
	
	\label{eq:}
\end{equation}

